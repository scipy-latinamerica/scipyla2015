\documentclass[11pt,a4paper]{report}
\usepackage[utf8]{inputenc}
\usepackage[portuguese]{babel}
\usepackage{graphicx}
\usepackage{hyperref}
\usepackage{enumitem}
\usepackage{fancyhdr}
\hypersetup{colorlinks=false,linkcolor=black,citecolor=black,hidelinks}
\renewcommand{\baselinestretch}{0.97}
\textwidth = 14cm
\pagestyle{fancy}

% Petición de @RTFMCelia
\lhead{
    \includegraphics[scale = 0.2]{./img/ugd.png} % imagen centro del pie
}

\begin{document}
% texto izquierda de la cabecera
\begin{center}
\chead{
    \small{Proposta de Patrocínio - SciPyLA 2015}
}
\end{center}

% Petición de @RTFMCelia
\rhead{
    \includegraphics[scale = 0.2]{./img/scipy2424.png}} % imagen centro del pie


\hfill \\ Posadas, 4 de Fevereiro de 2015\\[0.1cm]

\section*{REF: Proposta de Patrocínio}

Temos a honra de nos dirigirmos ao senhor/a senhora, como membros da Comissão Organizadora Local da
\textbf{3ª Conferência Latinoamericana de Python na Ciência}, com o objetivo de solicitar
à instituição que representas auxílio para este importante evento científico, que se realizará na cidade
de Posadas, Argentina, de 20 a 22 de Maio de 2015, na Universidad Gastón Dachary\footnote{\url{http://www.ugd.edu.ar/}}. \\[0.1cm]

O citado encontro consistirá de várias oficinas e palestras voltadas a
pesquisadores, professores de diferentes níveis educacionais,
estudantes, profissionais e empresários. \\[0.1cm]

Tomando-se por base os encontros anteriores realizados na Argentina e
Brasil, espera-se uma participação de aproximadamente 200 participantes
provenientes das mais diferentes regiões. Na conferência
palestrarão destacados pesquisadores de universidades Latinoamericanas e
do mundo. Para o momento contamos com a \emph{pré-confirmação} de
\textbf{Juan Luis Cano} e \textbf{Kiko Correoso}, autores de uma das
mais influentes fontes de informação sobre programação científica em espanhol,
autores do blog Pybonacci\footnote{\url{http://pybonacci.org/}}, além
de relevantes personalidades de nossa comunidade. Em anexo temos um documento
que explica com mais detalhes as características da
conferência a ser realizada.\\[0.1cm]

Por conta da Universidad Gastón Dachary, como
anfitriã, ter que organizar a Cerimônia de Abertura e auxiliar
na participação de destacados pesquisadores e da organização geral, haverá um
considerável investimento normalmente financiado com o aporte de
instituições públicas e privadas. Por isso, requeremos de vossa
instituição apoio financeiro para custear estes gastos, com a convicção
de que o êxito desta conferência é importante não apenas para a
Faculdade mas também para a comunidade educacional, científica e industrial, e
todos os demais influenciados por estas.\\[0.1cm]

Agradecemos desde já sua colaboração e o/a saudamos
com distinta consideração.\\[0.1cm]



\begin{flushright}
Eng. Juan B Cabral\\
Pela Comissão Organizadora Local \\

\end{flushright}
\newpage
\section*{Estrutura da Conferência}

    A conferência será voltada para a divulgação de ferramentas e/ou
    projetos realizados com Python no ambiente científico, acadêmico e
    industrial, e seus principais objetivos incluem:

  \begin{itemize}
    \item Dar oportunidade a divulgação da linguagem Python na comunidade científica Latinoamericana,
    \item Divulgar e/ou revisitar ferramentas disponíveis aplicadas a problemas científicos ou outras de natureza similar,
    \item Apresentar bibliotecas científicas desenvolvidas pela comunidade,
    \item Combinar educação, engenharia e ciência através de Python,
    \item Criar um marco apropriado para realização de encontros, fóruns de discussão, projetos educativos e de pesquisa relacionados a Python.
  \end{itemize}

A organização e desenvolvimento da conferência serão encaminhados
por um comitê de docentes dos Departamentos de Informática da UGD e
membros da comunidade SciPy Latin América.\\

O comitê organizador local está constituído pelos seguintes membros:\\

\begin{itemize}[nolistsep]
    \item Eng. Juan B Cabral (Scipy LA)
    \item Eng. Roberto Suenaga, (UGD)
    \item MSc. Filipe Saraiva (Scipy LA)
\end{itemize}

\textbf{Comitê acadêmico a confirmar.}

\section*{Planos de Patrocínio}
\begin{center}

\begin{tabular}{|p{4cm}|p{2.5cm}|p{2.5cm}|p{2.5cm}|p{2.5cm}|}
\hline
Benefícios / Categoria & Diamante & Ouro  & Prata & Bronze \\
\hline
Espaço no programa e descrição impressa/web & 1 página +
1000 palavras & 1/2 página + 200 palavras & 1/4 página + 100 palavras &  1/8 página + 50 palavras\\
\hline
Inserção de materiais na bolsa/pasta (previstos) & Sim & Sim & Sim & Sim \\
\hline
Logo nos \textit{souvenirs} & Sim & Sim & --- & --- \\
\hline
Logo/Link no site & Sim & Sim & Sim & Sim \\
\hline
Nome em Anúncios & Incluído & Incluído & --- & --- \\
\hline
Logo em pôster, caderno, folhetos & Extra-grande & Grande & Mediano & Pequeno \\
\hline
Banner em auditório & Triplo & Duplo & Simples & Simples \\
\hline
Autorização para oferta de empregos e acesso a e-mails/CV & Sim & Sim & Sim & Sim \\
\hline
Aporte (mínimo) & u\$s 3000 & u\$s 2500 & u\$s 1250 & u\$s 1000\\
\hline
\end{tabular}
\end{center}

\section*{Contato}
\noindent \href{mailto:jbc.develop@gmail.com}{jbc.develop@gmail.com}

\section*{Dados Bancários}
\noindent Banco de la Nación Argentina \\
Sucursal 2720-Posadas \\
\textbf{Denominación} Fundación IPESMI - UVT Gaston Dachary \\
\textbf{Cuenta corriente en pesos} Nº 4070079306 \\
\textbf{CBU} 01104077 - 20040700793066 \\
\textbf{CUIT} 30-68323774-0 \\

\end{document}
