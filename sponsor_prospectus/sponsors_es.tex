\documentclass[11pt,a4paper]{report}
\usepackage[utf8]{inputenc}
\usepackage[spanish]{babel}
\usepackage{graphicx}
\usepackage{hyperref}
\usepackage{enumitem}
\usepackage{fancyhdr}
\hypersetup{colorlinks=false,linkcolor=black,citecolor=black,hidelinks}
\renewcommand{\baselinestretch}{0.97}
\textwidth = 14cm
\pagestyle{fancy}

% Petición de @RTFMCelia
\lhead{
    \includegraphics[scale = 0.2]{./img/ugd.png} % imagen centro del pie
}

\begin{document}
% texto izquierda de la cabecera
\begin{center}
\chead{
    \small{Propuesta de Patrocinio - SciPyLA 2015}
}
\end{center}

% Petición de @RTFMCelia
\rhead{
    \includegraphics[scale = 0.2]{./img/scipy2424.png}} % imagen centro del pie


\hfill \\[0.2cm] Posadas, 4 de Febrero de 2015\\[0.4cm]

\section*{REF: Propuesta de Patrocinio}

Tenemos el agrado de dirigirnos a Ud. en nuestro carácter de miembros de la Comisión Organizadora Local de la
\textbf{3ra Conferencia Latinoamericana de Python en la Ciencia}, con el objeto de solicitar el auspicio de
la Institución que usted representa a tan importante evento científico, que tendrá lugar en la pintoresca ciudad
de Posadas la semana del 20 al 22 de Mayo de 2015, en la Universidad Gastón Dachary\footnote{\url{http://www.ugd.edu.ar/}}. \\

La citada Reunión consta de varios talleres y charlas  que convocan a
Investigadores, Docentes de los diferentes niveles educativos,
Estudiantes, Profesionales y Empresarios. \\[0.2cm]

Tomando en cuenta las anteriores reuniones similares en la Argentina y
Brasil, se espera una asistencia de aproximadamente 200 participantes,
provenientes de las distintas regiones. En las Reuniones
disertarán distinguidos científicos de Universidades de latinoamérica y
del mundo, por ahora contamos con la \emph{pre-confirmación} de
\textbf{Juan Luis Cano} y \textbf{Kiko Correoso} autores de una de las
mayores fuentes de información sobre programación científica en español
como lo es el blog Pybonacci\footnote{\url{http://pybonacci.org/}} así
como relevantes personalidades de nuestro medio. Adjuntamos un Anexo
que explica con más detalles las características de la
Reunión a realizar.\\[0.2cm]

Así mismo, la Universidad Gastón Dachary,  en su carácter de
anfitrión, tendrán que afrontar para la Ceremonia Inaugural, atención
de los asistentes más destacados y organización general, un
considerable gasto que tradicionalmente se financia con el aporte de
instituciones públicas y privadas del país. Por ello requerimos de
ustedes apoyo económico para solventar tales gastos, con la convicción
de que el éxito de esta Conferencia es importante no sólo para la
Facultad sino para la comunidad educativa, científica e industrial de
toda su zona de influencia.

Agradeciendo desde ya su notable colaboración, saludamos a Ud.
con distinguida consideración\\[0.5cm]


\begin{flushright}
Ing. Juan B Cabral\\
Por la Comisión Organizadora Local \\

\end{flushright}
\newpage
\section*{Estructura de la Conferencia}

    La conferencia está orientada a la divulgación de herramientas y/o
    proyectos realizados con Python en el ámbito científico, académico e
    industrial, y sus principales objetivos incluyen: Dar oportunidad a la
    divulgación del lenguaje Python en la comunidad científica Latinoamericana.
    Dar a conocer y/o revisar herramientas disponibles, orientadas a problemas
    científicos así como a otros de naturaleza más puntual.
  \begin{itemize}
    \item Mostrar bibliotecas científicas desarrolladas por la comunidad mundial.
    \item Combinar la educación, la ingeniería y la ciencia a través de Python.
    \item Generar un marco apropiado para concretar encuentros, foros de discusión y proyectos educativos y de investigación y desarrollo relacionados con Python.
  \end{itemize}
La organización y el desarrollo de la Conferencia serán llevados adelante
por un comité de docentes de los Departamentos de Informática de la UGD junto
con la comunidad SciPy Latin América.

Este comité organizador local está constituido por los siguientes docentes:\\
\begin{itemize}[nolistsep]
    \item Ing. Juan B Cabral (Scipy LA)
    \item Ing. Roberto Suenaga, (UGD)
    \item MSc. Filipe Saraiva (Scipy LA)
\end{itemize}

\textbf{Comité academico a confirmar.}

\section*{Planes de Patrocinio}
\begin{center}

\begin{tabular}{|p{4cm}|p{2.5cm}|p{2.5cm}|p{2.5cm}|p{2.5cm}|}
\hline
Beneficios / Categoría & Diamante & Oro  & Plata & Bronce \\
\hline
Espacio en el Programa y descripción impresa/web & 1 página +
1000 palabras & 1/2 página + 200 palabras & 1/4 página + 100 palabras &  1/8 página + 50 palabras\\
\hline
Inserción materiales en bolsa/carpeta (provistos) & sí & sí & sí & sí \\
\hline
Logo en regalo/recuerdo & sí & sí & --- & --- \\
\hline
Logo/Enlace en sitio web & sí & sí & si & si \\
\hline
Nombre en Anuncios & incluido & incluido & --- & --- \\
\hline
Logo en Afiche, Folletos, Volantes & extra-grande & grande & mediano & pequeño \\
\hline
Estandarte en auditorio & triple & doble & simple & simple \\
\hline
Oferta Empleos y acceso a emails/ CV autorizados  & sí & sí & sí & sí \\
\hline
Aporte (mínimo) & u\$s 3000 & u\$s 2500 & u\$s 1250 & u\$s 1000\\
\hline
\end{tabular}
\end{center}

\section*{Contacto}
\noindent \href{mailto:jbc.develop@gmail.com}{jbc.develop@gmail.com}

\section*{Datos Bancarios}
\noindent Banco de la Nación Argentina \\
Sucursal 2720-Posadas \\
\textbf{Denominación} Fundación IPESMI - UVT Gaston Dachary \\
\textbf{Cuenta corriente en pesos} Nº 4070079306 \\
\textbf{CBU} 01104077 - 20040700793066 \\
\textbf{CUIT} 30-68323774-0 \\


\end{document}
